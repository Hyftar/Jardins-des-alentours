\documentclass{scrreprt}
\usepackage{listings}
\usepackage{underscore}
\usepackage{graphicx}
\usepackage[bookmarks=true]{hyperref}
\usepackage[utf8]{inputenc}
\usepackage[french]{babel}

\usepackage{array}
\usepackage{cmap}

\renewcommand{\contentsname}{Table des matières}
\def\myversion{1.0}
\def\projectname{Jardins des Alentours}
\hypersetup{
    bookmarks=false,    % show bookmarks bar?
    pdftitle={Cahier des charges - \projectname},    % title
    pdfauthor={Simon Landry, Nicholas Massé & Keven Aubin}, % author
    pdfsubject={Cahier des charges pour jardins des alentours},  % subject of the document
    pdfkeywords={},        % list of keywords
    colorlinks=true,       % false: boxed links; true: colored links
    linkcolor=black,       % color of internal links
    citecolor=black,       % color of links to bibliography
    filecolor=black,       % color of file links
    urlcolor=blue,         % color of external links
    linktoc=page           % only page is linked
}

\usepackage{hyperref}
\begin{document}
\pagenumbering{gobble}
\begin{flushright}
 \rule{16cm}{5pt}\vskip1cm
 \begin{bfseries}
  \Huge{CAHIER DES \\ CHARGES}\\
  \vspace{1.5cm}
  pour\\
  \vspace{1.5cm}
  \projectname\\
  \vspace{1.5cm}
  \LARGE{Version \myversion}\\
  \vspace{1.5cm}
  Préparé par :\\
  Simon Landry\\
  Keven Aubin\\
  et Nicholas Massé\\
  \vspace{1cm}
  Pour Programmation Breadsticks\\
  \vspace{.5cm}
  Février 2020\\
 \end{bfseries}
\end{flushright}
\newpage
\pagenumbering{roman}
\tableofcontents
\newpage
\pagenumbering{arabic}
\chapter{Présentation du projet}

\projectname{} se veut être une éloge informatisée, entièrement destinée aux végétaux.
Née d'une passion pour la nature, il s'agira d'un médium par lequel tous et chacun
pourront passer pour obtenir à la fois des informations utiles et du divertissement, tout
en ayant la possibilité d'y faire du commerce.

\paragraph{}
Construit sur les bases d'une application web, le projet sera offert sur ordinateur de bureau, tablette, et téléphone
et offrira la possibilité aux visiteurs de consulter, page par page, une myriade d'informations
liées aux fruits et aux légumes de son choix. Les plus actifs pourront échanger sur les méthodes de culture de même que contribuer à l'agrandissement de la base de données.
Les amateurs de produits hyperlocaux auront même accès à un panel renouvelé de jardiniers locaux
vendant le produit en question.

\paragraph{}
Dans la même optique, \projectname{} se veut être un accès facile, autant
pour les consommateurs que pour les cultivateurs, à l'échange et à l'achat de fruits et légumes.
Dans le but de favoriser la consommation et la culture de produits locaux, la plateforme fournira,
aux utilisateurs inscrits, des outils pour planifier, produire et vendre leurs produits. Les jardiniers
en herbe auront évidemment à leur disposition toutes les informations sur les fruits et légumes de
leur choix, fournies et centralisées par \projectname{}.

\paragraph{}
L'équipe derrière \projectname{} cultive aussi l'espoir que leur projet
créera des liens solides à travers la population en générale, mais surtout à
petite échelle, dans les quartiers résidentiels, là où, aujourd'hui, on ne retrouve
en majorité que les éternels gazons verts, on pourra se promener et observer
un nouvel écosystème avec une flore resplendissante.

\chapter{Objectifs et environnements}
\section{Objectifs}
Le point fort de \projectname{} passe avant tout par l'accessibilité.
S'il veut se démarquer de ses concurrents et élargir son auditoire, le projet
devra intégrer un accès et une navigation aisée de même que vêtir une interface visuellement séductive.
Il s'agit d'un facteur de la plus haute importance considérant que plusieurs aspects
de l'application en dépendent, en particulier l'envergure de l'information disponible par rapport aux fruits et légumes,
le flux d'interactions entre les utilisateurs et surtout la disponibilité des produits locaux, qui existe grâce aux jardiniers
inscrits à l'application.

\paragraph{}
Une interface d'administration destinée à la fois aux administrateurs du site
ainsi qu'aux contributeurs du site devra être mise en place pour que le projet puisse profiter
d'une meilleure stabilité dans son développement à long terme. Celle-ci devra
être instinctive et facile d'utilisation afin de s'assurer que les jardiniers
contributeurs n'aient pas besoin de suivre une formation afin d'y contribuer.


\section{Environnements}
La mise en place de l'application web inclut trois branches
principales: les environnements «Visiteur», «Jardinier» et «Administrateur».

\paragraph{}
L'environnement «Visiteur» est celui destiné à l'utilisateur client du site, qu'il soit inscrit ou non.
Il s'agit de l'environnement le plus vaste, et donne accès à presque toute la plateforme, c'est-à-dire les multiples pages
de produits, les questions et réponses et les jardiniers locaux. Lorsqu'il est inscrit, l'utilisateur pourra
aussi échanger, commenter et discuter du légume en question.

\paragraph{}
L'environnement «Jardinier» permettra à un utilisateur, après sa soumission à certaines vérifications,
d'accéder à une différente interface destinée à la culture et à la vente de ses fruits et légumes. Celle-ci
inclura entre autres un étalage virtuel ainsi, qu'à l'avenir, divers outils tels qu'un calendrier de semis et un planificateur de jardins.

\paragraph{}
L'environnement «Administrateur» permettra à l'administrateur ou son mandataire
de gérer le contenu mis en ligne et d'effectuer la modération par rapport aux utilisateurs inscrits.


\chapter{Ressources allouées}
\section{Financement et matériel}

Pour \projectname{}, il ne s'agit que du début d'une aventure. Or, si le temps imparti
pour le noyau de l'application est défini dans une limite temporelle de 5 semaines, la volonté de
\projectname{} est de garder en tête son extensibilité puisque ses développeurs auront l'occasion de poursuivre
son développement dans le futur.

\paragraph{}
De la même manière, le budget alloué pour l'instant au développement de l'application n'est que celui des ressources technologiques de base utilisées
par les développeurs.

\section{Équipe de développement}

L'équipe de développement est composée de trois développeurs en herbe terminant leur formation de techniciens en informatique de gestion.
Forts de quelques projets accomplis dans le passé, leurs intérêts communs les réunissent une fois de plus pour
mener à bien le développement de cette nouvelle application web.

\paragraph{}
Ci-dessous se trouve un petit descriptif de chacun des membres de cette équipe.

\subsection{Simon Landry}

Simon Landry est un développeur back-end terminant sa formation au Collège de
Maisonneuve. Son expertise en développement sur divers projets «Open Source»,
qui s'échelonne sur plus d'une dizaine d'années, lui permettra de conseiller
ses collègues sur les meilleures pratiques de développement et les multiples
outils qui entourent le développement.

\paragraph{}
Expérimenté à la tâche de Scrum Master, Simon saura mener l'équipe dans
ce rôle lors des scrums journaliers et saura protéger l'équipe des
obstacles et distractions qui l'entourent. Après une année au sein des
Forces Armées Canadiennes en tant qu'officier, ce développeur saura utiliser cette
expérience en leadership pour mener le projet à terme.

\newpage

\subsection{Keven Aubin}

Keven Aubin est un développeur web terminant sa formation au Collège de
Maisonneuve. Son expérience en développement front-end et back-end en
feront un membre polyvalent de l'équipe.
Titulaire d'un diplôme de droit notarial, ce développeur utilise son
expérience de la communication pour mieux communiquer avec le client et
ainsi comprendre plus facilement les besoins et désirs du maître
d'ouvrage.

\subsection{Nicholas Massé}

Issu du domaine de la restauration et de la sommellerie, Nicholas Massé finit
présentement ses études en développement web au Collège de Maisonneuve et occupe également
un poste de développeur dans le domaine du marketing en ligne.

\paragraph{}
Dans une plus grande mesure, Nicholas a effectué par le passé des voyages à
l'étranger qui sont étroitement liés à l'agriculture biologique.
Il s'adonne annuellement au jardinage sur son propre petit bout de lopin à
la campagne et possède un savoir-faire indéniable dans ce domaine. Il
sera donc une référence importante tout au long du projet lorsqu'il s'agira de
bien définir les informations pertinentes concernant les produits et les intérêts
des jardiniers de la plate-forme.

\chapter{Besoins}
Dans une première mesure, il est impératif, pour atteindre nos objectifs d'accessibilité,
de rendre l'application disponible sur les principaux dispositifs d'accès à
l'Internet, c'est-à-dire
l'ordinateur de bureau, la tablette et le téléphone intelligent. Il est évident que
notre application devra également s'adapter aux navigateurs web les plus populaires.

\paragraph{}
Dans le même ordre d'idées, \projectname{} devra être disponible dans un nombre
maximal de langues afin de former des communautés de jardiniers
provenant des quatre coins du monde, le tout dans le but de créer
un contenu riche, de multiplier les échanges et de stimuler les communautés de jardiniers.

\paragraph{}
Au sujet de l'interface destinée aux visiteurs, elle devra rassembler en un endroit principal
l'accès aux divers modules de l'application, tout en gardant à l'honneur
le produit de la page en question. En naviguant vers ces modules,
les visiteurs auront accès à des interfaces ultérieures, leur permettant d'obtenir
des détails sur la culture du légume, l'interface de recherche de jardiniers locaux ainsi que l'accès
à un système de conversations portant sur le légume en particulier.

\paragraph{}
Pour pallier à la complexité structurelle de l'application, \projectname{} incorpore
le concept de communautés, qui permettra aux utilisateurs d'identifier à l'avance leur préférence
en terme de géolocalisation et de produits. De cette manière, l'application
personnalise l'expérience de l'utilisateur et approfondit son sentiment d'appartenance.

\paragraph{}
Lorsqu'inscrit, un utilisateur doit pouvoir consulter et modifier son profil et
avoir accès à un maximum d'informations reliées à l'historique de ses activités.

\paragraph{}
L'interface du jardinier devra permettre à ce dernier de publier sur notre site
les légumes qu'il désire mettre en vente via un étalage virtuel. Cet étalage lui
demandera une certaine quantité d'information destinée à la vente du produit, qui lui devra être
choisi au moyen d'une liste de variétés de fruits ou légumes, en synchronisation avec ceux qui
apparaissent sur le site.

\paragraph{}
L'interface d'administration, quant à elle, devra permettre l'ajout de produits
et variétés, à l'aide d'un système de contribution par les utilisateurs.
L'administrateur aura également accès à une gestion sommaire des visiteurs,
des jardiniers et des contributeurs.


\chapter{Contraintes}
Pour parvenir à construire l'écosystème informatique relativement complexe qu'incarne \projectname{},
l'équipe de développement a choisi d'utiliser Ruby comme langage de programmation principal et Ruby on Rails comme framework principal.
Ce choix a également été fait dans le but d'optimiser le temps de réalisation du projet.

\paragraph{}
Comme l'application nécessite la manipulation d'une bonne quantité d'information
et que celle-ci peut s'avérer complexe, PostgreSQL a semblé
s'imposer comme choix de base de données en raison de sa licence libre et de sa
gratuité en terme de coûts. PostgreSQL est également un système de gestion de
base de données qui s'incorpore très bien avec Ruby on Rails.

\chapter{Fonctionnalités}

\section{Interfaces}
\begin{tabular}{|l|p{9cm}|r|}
 \hline
 \bfseries Code & \bfseries Fonctionnalité                             & \bfseries Réalisation \\
 \hline
 I1-1           & Page d'accueil                                       &                       \\
 \hline
 I2-1           & Affichage des communautés                            &                       \\
 \hline
 I2-2           & Questions et réponses d'une communauté               &                       \\
 \hline
 I2-3           & Discussions d'une communauté                         &                       \\
 \hline
 I2-4           & Géolocalisation de la communauté sur une carte       &                       \\
 \hline
 I3-1           & Affichage des pages des jardiniers                   &                       \\
 \hline
 I3-2           & Affichage des jardins et produits                    &                       \\
 \hline
 I3-3           & Affichage des marchés des jardiniers                 &                       \\
 \hline
 I3-4           & Géolocalisation des jardins et marchés sur une carte &                       \\
 \hline
 I4-1           & Signalement des visiteurs malveillants               &                       \\
 \hline
 I4-2           & Signalement des jardiniers malveillants              &                       \\
 \hline
\end{tabular}


\section{Administration}
\begin{tabular}{|l|p{9cm}|r|}
 \hline
 \bfseries Code & \bfseries Fonctionnalité                                 & \bfseries Réalisation \\
 \hline
 A1-1           & Gestion des communautés                                  &                       \\
 \hline
 A1-2           & Gestion des informations des produits et variétés        &                       \\
 \hline
 A1-3           & Gestion des jardiniers, administrateurs et contributeurs &                       \\
 \hline
 A2-1           & Suspension de visiteurs malveillants                     &                       \\
 \hline
 A2-2           & Suspension de jardiniers malveillants                    &                       \\
 \hline
\end{tabular}

\section{Navigation}
\begin{tabular}{|l|p{9cm}|r|}
 \hline
 \bfseries Code & \bfseries Fonctionnalité              & \bfseries Réalisation \\
 \hline
 N1-1           & Barre de navigation version mobile    &                       \\
 \hline
 N1-2           & Barre de navigation version bureau    &                       \\
 \hline
 N2-1           & Navigation par recherche              &                       \\
 \hline
 N3-1           & Aide à l'utilisation de la plateforme &                       \\
 \hline
\end{tabular}
\section{Sécurité}
\begin{tabular}{|l|p{9cm}|r|}
 \hline
 \bfseries Code & \bfseries Fonctionnalité                                                                           & \bfseries Réalisation \\
 \hline
 S1-1           & Application des techniques de validation de données                                                &                       \\
 \hline
 S2-1           & Protection des informations du compte administrateur à l'aide d'un système de rôles et permissions &                       \\
 \hline
\end{tabular}
\section{Format}
\begin{tabular}{|l|p{9cm}|r|}
 \hline
 \bfseries Code & \bfseries Fonctionnalité                                          & \bfseries Réalisation \\
 \hline
 F1-1           & Conception de type MVC (\textit{Model View Controller})           &                       \\
 \hline
 F2-1           & Hébergement du site web                                           &                       \\
 \hline
 F2-2           & Hébergement de la base de données centralisée                     &                       \\
 \hline
 F3-1           & Données minimales nécessaires à l'optimisation du contenu du site &                       \\
 \hline
\end{tabular}

\chapter{Risques et priorités du projet}

Les risques du projet \projectname{} ont été limités à un niveau raisonnable.
Par contre, les technologies utilisées pour le projet sont nouvelles pour certains
membres de l'équipe de développement. Ainsi, l'équipe sera confrontée à l'apprentissage
de celles-ci dans un court délai.

\paragraph{}
Également, dans le passé, les membres de cette équipe ont fait preuve d'ambition
démesurée face à la réalisation de projets ayant pu être simplifiés de manière significative.
Dans le cadre de ce projet, afin de s'assurer que les fonctionnalités désirées
soient réalisées dans les délais prévus, une d'analyse de l'essentiel sera
primordiale.

\paragraph{}
Finalement, puisque les horaires de disponibilités des membres de l'équipe diffèrent,
une importance particulière devra être accordée à la communication entre les membres
de l'équipe afin de s'assurer que chacun soit informé de l'état d'avancement des
diverses fonctionnalités.

\chapter{Calendrier des livrables}
\begin{tabular}{|l|p{9cm}|l|}
 \hline
 \bfseries Livrable & \bfseries Description  & \bfseries Date de fin \\
 \hline
 Étape 1            & Cahier des charges     & 23 Février 2020       \\
 \hline
 Étape 2            & Analyse                & 1 Mars 2020           \\
 \hline
 Étape 3            & Prototype              & 8 Mars 2020           \\
 \hline
 Étape 3            & Développement et tests & 20 Mars 2020          \\
 \hline
 Étape 4            & Documentation          & 20 Mars 2020          \\
 \hline
\end{tabular}

\appendix

\chapter{Fonctionnalités potentielles}
Afin de respecter les délais alloués, un effort de sobriété a été fait dans
l'élaboration du projet \projectname{}. Ainsi, certaines fonctionnalités,
bien qu'intéressantes, ont été mises sur la glace le temps d'assurer le
développement des fonctionnalités principales. Advenant la livraison du projet
tel que présenté avant la date du 20 mars, certaines autres fonctionnalités
pourraient être implémentées.

\paragraph{}
Diverses recettes pourraient être suggérées au sein des communautés. Ainsi,
les produits disponibles sur la plateforme pourraient être mis en valeur.
Cette fonctionnalité pourrait attirer de nouveaux visiteurs sur la plateforme
et ainsi augmenter la visibilité de l'offre de produits disponibles dans les
marchés.

\paragraph{}
Au sein des communautés, une discussion en temps réel sous forme de clavardage
permettra d'augmenter le nombre d'interactions entre les jardiniers. Ceci
créera un plus grand sentiment d'appartenance à la plateforme pour les jardiniers.

\paragraph{}
Un système de détection automatique de la zone de rusticité des jardins selon la
géolocalisation permettra aux jardiniers novices d'obtenir automatiquement des informations
adaptées à leur environnement de culture. Ils pourront ainsi connaître automatiquement
quels produits peuvent être cultivés dans leur jardin et à quel moment de l'année
la culture de ces derniers doit être commencée.

\chapter{Répertoires de travail}
Adresse du répertoire Github : \textbf{https://github.com/Hyftar/Jardins-des-alentours.git}

\end{document}
